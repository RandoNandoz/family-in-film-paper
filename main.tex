\documentclass[12pt, letterpaper]{article}
\usepackage[letterpaper, margin=1in]{geometry}
\usepackage{unicode-math} % loads fontspec automatically
\usepackage{setspace}
\usepackage[hidelinks]{hyperref}
\usepackage[american]{babel} 
\usepackage{csquotes} 
\usepackage[style=mla,hyperref=true]{biblatex}

% shut up the horizontal box overflows
\sloppy


\doublespacing
\setmainfont{Times New Roman}

\addbibresource{refs.bib}

\title{Something to do with modern family}
\author{Randy Zhu}
\date{2023-03-04}

\begin{document}

\begin{titlepage}
    \maketitle
\end{titlepage}

% begin actual text

\begin{center}
    \textbf{Preface}
\end{center}

\textit{Modern Family} is an American TV show by Christopher Lloyd and Steven Levitan. We're introduced to three families: A skip-generation, stepfamily, consisting of Jay Pritchett, his Columbian immigrant wife whom Jay re-married, Gloria Pritchett (née Delgado), and their child, Manny Delgado. A same-sex family, with Mitchell Pritchett, Cameron Tucker, and their adopted daughter, Lily Tucker-Pritchett. Lastly, a traditional, nuclear family, with Claire Dunphy (née Pritchett) and Phil Dunphy as well as their children. The show outlines the similarities and differences between the families, expounding what it means to be in a \textit{modern} family. All families in the show are tied together by Jay Pritchett, a wealthy patriarch whose children, Gloria and Mitchell, are members of their own respective families in the show. In this paper, we'll focus primarily on Mitchell's family, and at times compare with the other families in the show.

\begin{center}
    \textbf{Vanier Institute Functions of a Family Connection}
\end{center}

Mitchell and Cameron in the episode ``\citetitle{modern_family_pilot:2009}'' provide for the ``addition of new members [of the family] through procreation or adoption'', one of the Vanier Institute functions of a family, \autocite[11]{mitchell:2018} by adopting Lily Tucker-Pritchett from Vietnam. Mitchell and Cameron, despite not being able to have biological children themselves as they are both cis-gendered men, still provide for this function by adoption. We still see examples of a heteronormative and homophobic society, as Mitchell is told by a man that ``[Lily]'s an angel, you and your wife must be thrilled'' (``\citetitle{modern_family_pilot:2009}'' 2:50). When Cameron shows up, the man begins to glare, perhaps casting judgment on the homosexual couple raising a child. According to \autocite[239-240]{mitchell:2018} these judgments are unfounded, considering the lack of empirical evidence that harm is done to children raised by same-sex couples. This scene in the first episode outlines how this specific same-sex family fulfills the function of family and the struggles that same-sex couples may face when raising and socializing their children.


In the episode ``\citetitle{modern_family_fizbo:2009}'', Cameron demonstrates the ``physical maintenance and care of [family] group members'' \autocite[11]{mitchell:2018},
dressed as Fizbo the Clown defends his boyfriend and fellow family member, Mitchell. As \autocite{mirabelli2018s} explains, the family functions described by the Vanier Institute are ``[deliberately] broad'' \autocite[3]{mirabelli2018s}, which means that what counts as ``physical maintenance and care of [family] group members'' is open to interpretation. Actions like, but are not limited to: tending to family members when they are ill, providing advice, and emotional support, and in our case, defending our family members and putting ourselves in harm's way for them. In this episode when another man bumps Mitchell with his truck, Mitchell, having a more frail and smaller appearance confronts the man when he is hit, but the man escalates the situation and physically alterates with Mitchell. Cameron sees this in his car, and intervenes, verbally and physically threatening the man, using language like, ``I will twist you like a balloon animal'' and ``I will beat your head against this bumper until the airbags deploy'' (``\citetitle{modern_family_fizbo:2009}'' 10:20), as wells as some rough taps on the shoulder. Cameron in this instance is caring for his fellow family member, refusing to let his boyfriend get disrespected and struck by a negligent driver.

\begin{center}
    \textbf{Applying a Conflict Perspective}
\end{center}

According to \cite{mitchell:2018}, conflict theory looks at families from both a micro level, of intra-family conflict, and a macro level, between families and elements of society, like the workforce, or economic conflicts. From the conflict theory perspective, conflicts are not necessarily bad things that occur between families, but instead a consequence of society and family (31-33).

When approaching this episode with a conflict theory, we see both a racial and gender inter-family conflict in the episode titled ``\citetitle{modern_family_fears:2009}''. When Mitchell and Cameron invite Lily's Pediatrician, Miura to have brunch with them, Lily says ``Mommy'' when she is being held by Miura, and this triggers an emotional response from Cameron (6:25). Cameron laments ``maybe we're not providing her with the feminine energy she needs'', which demonstrates a conflict with the inequality and perceived roles that men and women should play in society. These societal conflicts in turn also cause inter-family conflicts between the family and their new friend, when Mitchell goes on to try to justify Cameron's perspective by explaining that it's because Miura is Asian and that Lily was simply ``used to Asians'' (10:45). These conflicts make the conversation awkward, and cause them some emotional distress (11:05).

At the end of the episode, we see a critical point of conflict theory in action: conflict is not necessarily bad. Even though these conflicts cause distress at first, the resolution of these conflicts sometimes leads to strengthened family bonds. At the end of the episode, Mitchell and Cameron apologize for their behavior and their comments (16:30), but Miura explains that she had a difficult and complex relationship with her mother, and how she was closer to her father. She goes on to tell the couple that ``having two fathers that care as much as you do makes Lily the luckiest girl in the world'' (17:00). This likely strengthened the bond between Mitchell and Cameron, even though there was conflict.

We can trace the origins of conflict theory to Karl Marx's work on class, capitalism, and exploitation \autocite[32]{mitchell:2018}, and we can see an instance of how class exploitation affects the family in the episode ``\citetitle{modern_family_tbt:2009}''. In the episode, we see a case of a conflict between classes that challenges the unity of families when Mitchell is called into work (2:30). Mitchell's resentment towards his boss eventually boils over, and he quits. From a Marxian perspective, Mitchell's boss is a member of the bourgeoisie, as he is a senior partner of the legal firm, and he owns a stake in the firm, while Mitchell is an associate with no stake. This is a clear example of class conflict between Mitchell and his boss. When Mitchell quits, while he is elated at first when he quits, the stress of having no income whilst raising a child stresses him out further. We see an example of macro-level class conflict between a family's breadwinner and the capital class.

\begin{center}
    \textbf{Connections to Sexuality and Class}
\end{center}

Sexuality, from a sociological perspective, is not solely sexual orientation, but dependent on a sexuality trichotomy consisting of sexual orientation, sexual behavior, and sexual identity \autocite{sexuality_lecture}. Sexual orientation is who you are attracted to, sexual behavior is who you have sex with, and identity is who you tell others or yourself you are attracted to. All these factors are distinct, and fluid, as we will see with examples from Modern Family.

In ``\citetitle{modern_family_the_kiss:2009}'', we see the distinction and fluidity of sexual behavior and expression when concerns arise from the lack of affection that Mitchell shows Cameron in public (1:36). While at the end of the episode, their extended family attributes it to Jay, Mitchell's father not showing his children enough affection, it is also possible that due to negative societal perceptions of LGBTQ+ individuals, that causes Mitchell's lack of affection to Cameron in public. This hypothesis is supposed by \autocite{stammwitz2021public} who found that LGBTQ+ individuals enjoy public displays of affection less in places where they feel less supported.

We see an example of the perception of actions taken by LGBTQ+ individuals as inherently sexual that was quoted in \autocite{sexuality_lecture} in the episode ``\citetitle{modern_family_fears:2009}''. When Cameron plays racquetball with his father-in-law, Jay, Jay makes comments telling Cameron that ``[there are] showers [that] are private'' suggesting that Cameron might somehow sexualize other men in the locker room. This also contributes to the challenges outlined by \autocite{sexuality_lecture}, namely how LGBTQ+ individuals face challenges forming relationships. If individuals perceive mundane activities like showering in a public gym as sexual, they may hesitate when forming new relationships with LGBTQ+ individuals.

Social class is difficult to define, and there are differing perspectives on it. Bourdieu's social capital perspective on social class defines social class as the amount of social capital --- one's ``access to resources [...] because of who they know'' \autocite{social_class_lecture}. In ``Modern Family'', all the families are upper-middle class and are heavily supported by their social capital, as all of the families are tied together by Jay, a wealthy businessman who has a significant amount of connections.

% Mitchell being a ``nepo-baby''
We can see Mitchell leverage his own, and his father's social capital, by leveraging his father's connections in the business world after he quits his job in ``\citetitle{modern_family_moon_plugged:2009}''. As he discovers that he is not fit to be a stay-at-home parent, and instead serves a better role as a breadwinner. When he quit his job previously, he burned his bridges by resigning without notice and ridiculing his boss (``\citetitle{modern_family_tbt:2009}''). Normally, this would impede one's career, but with his father's support, the social capital he has moves him upwards in his career. Even from the beginning of his career, using a Weberian life-chance perspective, the wealth of his father, his financial support, and not having to be concerned financially youth allowed him to attend a prestigious Ivy League school and later law school, which already put him ahead in terms of social class.

% Mitchell knows someone to get his daughter into preschool
% Mitchell also has non-career social capital. As he is trying to get Lily into preschool, he calls up his sister Claire, who has connections with the local Wagon Wheel preschool (``\citetitle{modern_family_moon_plugged:2009}'' 1:35).
Previously, Mitchell and Cameron both agreed to wait one more year until Lily went to preschool, but when the children they interact with in a similar social class to them all go to preschool, they choose to follow what the other parents do. We can interpret these actions from both Weberian and Bourdieuian lenses. Using Weber's perspective that social class is dictated by your opportunities to achieve goals, Mitchell and Cameron are both trying to increase Lily's opportunities compared to her peers, placing her in school earlier, regardless of motive still exposes her to more people, and in turn more chances in her life, and increasing her chances for success. This also agrees with the Bordieuian lens of looking at the situation. As all of Lily's peers depart, naturally, there are fewer other kids for her to grow up with and stay connected to, thus, Mitchell and Cameron view this as a loss of potential social capital, and so they try to place her into preschool to restore that lost social capital.

Lastly, we can see that Bourdieu's social capital is not limited to just professional networking, as Mitchell calls in her sister who has connections to preschools to not only get him a spot for Lily but also gain knowledge about preschools in the area from a previous parent.


% similarity with my family




% differences

% begin bibliography


\clearpage

\printbibliography

\end{document}